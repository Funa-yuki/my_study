\documentclass[a4paper,12pt]{jreport}
\usepackage{jgraduate}

\title{Webアプリケーションを安全に\\するフレームワークの新しい機能}
\author{久保田 康平}
\university{九州大学}
\department{大学院システム情報科学府}
\major{情報知能工学専攻}

\date{\today}

\begin{document}

\maketitle

\begin{abstract}
Webアプリケーション開発者が実装するコードを実行時に自動的に解析し,必要ならば修正する機能をWebアプリケーションフレームワークに持たせることを提案し,実装して評価を行う.
Webアプリケーションはインターネットを通して世界中から誰でも接続でき,対話的に通信できるという特徴から様々な攻撃の対象になる.
また,インターネットの普及に伴いWebアプリケーションの重要性は増し,同様にWebアプリケーションの防御もまた重要になっている.
脆弱性攻撃は,Webアプリケーションの設計上の欠点や仕様上の問題点である脆弱性を利用する攻撃である.
脆弱性攻撃の対策の一つは,Webアプリケーションに脆弱性を作らないことであり,そのためWebアプリケーション開発者はWebアプリケーションフレームワークを利用することがある.
Webアプリケーションフレームワークは,Webアプリケーション開発において利用することが多いメソッドを持つライブラリである.
それらのメソッドを利用することで効率よくアプリケーションを開発することができる.
セキュリティ面において,Webアプリケーションフレームワークが提供するメソッドは脆弱性対策がなされているものが多い.
したがって,Webアプリケーションフレームワークを利用した方が,利用しない時と比較して効率的にセキュアなWebアプリケーションを開発しやすい.
一方で,開発者は常に完全にセキュアなコードを書くことはできないため,Webアプリケーションフレームワークを利用して,脆弱性があるWebアプリケーションを実装してしまうことがある.
その理由の一つが,Webアプリケーション開発者がWebアプリケーションフレームワークを適切に利用できないことである.
Webアプリケーション開発者が,フレームワークのメソッドが持つセキュリティ機能を正しく理解していなかったり,
セキュリティ機能を持つメソッドを知らなかったりすることによって脆弱なWebアプリケーションが実装される.
この問題に対して本論文では,Webアプリケーション開発者が実装したソースコードを修正する機能を持つWebアプリケーションフレームワークを提案する.
提案手法を実証し評価を行った結果,この機能は実装されたコードの脆弱性を一部修正でき,レスポンスタイムは提案手法を適用しなかった場合とほとんど変わらないことを確認した.
実装された修正関数の蓄積は将来のアプリケーションのセキュリティの向上に寄与できるものである.
\end{abstract}

\tableofcontents
\listoffigures
\listoftables
\newpage
\pagenumbering{arabic}


\chapter{はじめに}
はじめに。はじめに。はじめに。はじめに。はじめに。はじめに。
はじめに。はじめに。はじめに。はじめに。はじめに。はじめに。
はじめに。はじめに。はじめに。はじめに。はじめに。はじめに。
はじめに。はじめに。はじめに。はじめに。はじめに。はじめに。
はじめに。はじめに。はじめに。はじめに。はじめに。はじめに。
はじめに。はじめに。はじめに。はじめに。はじめに。はじめに。
はじめに。はじめに。はじめに。はじめに。はじめに。はじめに。
はじめに。はじめに。はじめに。はじめに。はじめに。はじめに。
はじめに。はじめに。はじめに。はじめに。はじめに。はじめに。
はじめに。はじめに。はじめに。はじめに。はじめに。はじめに。
はじめに。はじめに。はじめに。はじめに。はじめに。はじめに。
はじめに。はじめに。はじめに。はじめに。はじめに。はじめに。
はじめに。はじめに。はじめに。はじめに。はじめに。はじめに。
はじめに。はじめに。はじめに。はじめに。はじめに。はじめに。
はじめに。はじめに。はじめに。はじめに。はじめに。はじめに。
はじめに。はじめに。はじめに。はじめに。はじめに。はじめに。
はじめに。はじめに。はじめに。はじめに。はじめに。はじめに。
はじめに。はじめに。はじめに。はじめに。はじめに。はじめに。
はじめに。はじめに。はじめに。はじめに。はじめに。はじめに。
はじめに。はじめに。はじめに。はじめに。はじめに。はじめに。
はじめに。はじめに。はじめに。はじめに。はじめに。はじめに。
はじめに。はじめに。はじめに。はじめに。はじめに。はじめに。
はじめに。はじめに。はじめに。はじめに。はじめに。はじめに。
はじめに。はじめに。はじめに。はじめに。はじめに。はじめに。
はじめに。はじめに。はじめに。はじめに。はじめに。はじめに。

\chapter{背景}
\section{最初}

\chapter{関連研究}
\section{論文1}
\section{論文2}
\section{論文3}

\chapter{提案手法}

\chapter{実装}

\chapter{実験}

\chapter{結果}

\chapter{考察}

\chapter{おわりに}


\end{document}
