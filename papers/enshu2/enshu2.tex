\documentclass[twocolumn, 10pt, a4paper]{jsarticle}
\usepackage[dvipdfmx]{graphicx}
\usepackage{url}
\usepackage{listings}
\lstset{
  frame=single,
  breaklines=true,
  numbers=left,
  numbersep=5pt,
  numberstyle=\small,
  language=python,
  tabsize=2,
  showstringspaces=false,
}
\usepackage{caption}
\usepackage[name=Fig]{bicaption}

%title
\title{Webアプリケーションを防御する新しい機能}
\author{久保田 康平}
\date{2020年8月11日}

\begin{document}
\maketitle
\section{はじめに}
本論文は,Webアプリケーションのセキュリティ機能向上を目的にしている.
その目的の達成のために,アプリケーション開発者が実装したプログラム中の関数や引数を
解析し,実行時にその関数に脆弱性があった時には修正することができるWebアプリケーションフレームワークを提案,実装し評価する.

Webアプリケーションフレームワークはアプリケーション開発者に,Webアプリケーションのセキュリティを向上させる機能を提供している.
したがって,アプリケーション開発者はWebアプリケーションフレームワーク上の関数を用いることで,脆弱性が低減されたWebアプリケーションを実装できる.
しかし,アプリケーション開発者は時としてWebアプリケーションフレームワークを利用して,脆弱性があるアプリケーションを実装することがある.
アプリケーション開発者が,Webアプリケーションフレームワークが提供する関数やモジュールを,適切に利用できないことがあるためである.

% Webアプリケーションフレームワーク自動防御
上記の問題に対してWebアプリケーションフレームワークの自動防御手法が提案されている.
J. Weinbergerらの論文\cite{weinberger2011systematic}では,クロスサイトスクリプティング脆弱性(XSS)の対策としてWebアプリケーションフレームワークが自動でクライアントの入力を無害化する自動サニタイズについて書かれている.
自動サニタイズによってWebアプリケーション開発者はXSSを対策する実装をすることなく,XSSの対策を適切に行うことが可能になる.
一方で自動サニタイズの多くはXSSやSQLインジェクション(SQLi)の対策である.
したがって多くのWebアプリケーションフレームワークは認可制御不備の脆弱性を自動的に防御する機能を持っていない.

Webアプリケーションの脆弱性を低減する手法の研究として,Webアプリケーションファイアウォール(WAF)の研究\cite{kruegel2003anomaly, epp2017anomaly}がある.
WAFは,XSSやSQLiなどのWebアプリケーションの脆弱性を利用する攻撃からWebアプリケーションを保護するシステムである.
Webアプリケーションとクライアント間の通信を監視し,攻撃である通信を検出した時,その通信を遮断したり修正したりする.
WAFは,アプリケーションを修正することなく,脆弱性に対する攻撃を低減することができるため,アプリケーションを直接修正できない状況の時に,特に有効である.
一方でWAFでは対策しづらい攻撃が存在する.
WAFはリクエスト内の特殊文字によって攻撃かどうかを判断するが,主に認証や認可について脆弱性があるWebアプリケーションに対する攻撃は,特殊文字を利用することなく行うことができるため,WAFでは防御しづらい.

これらの問題を解決するために,本論文ではアプリケーション開発者が実装したソースコードを修正し,必要に応じて修正するWebアプリケーションフレームワークが備えるべき機能を提案する.
アプリケーションの実行時に,アプリケーション開発者が実装したコールバック関数中に存在する脆弱な関数を安全な関数に変換したり,その関数の引数に対して入力検証を行う関数を自動で挿入したりすることで,脆弱性を低減するものである.
この機能はアプリケーション開発者が,脆弱性の解析のために追加の実装を行うことなく,アプリケーションを動的に検証する関数や変数を追加することができる.
ここでのコールバック関数とはアプリケーション開発者が実装し,Webアプリケーションフレームワークに格納される関数である.
この関数はパスやメソッドに紐づけられていてパスやメソッドに応じてフレームワークから呼び出される.
% 何ができるようになるか?認証、認可
コールバック関数を修正する機能により,これまでのWebアプリケーションフレームワークやWAFでは対策がしづらかった,管理者しか利用できない関数やデータについての脆弱性の対策をフレームワークが行うことができる.
権限が必要なメソッドをコールバック関数内で呼び出した時に,フレームワークが,リクエスト情報を利用して,正しい権限を持っているかどうかを検証するメソッドをコールバック関数内に追加してくれることが可能だからである.
加えて,この機能は実行時にアプリケーションを自動で修正するので,アプリケーション内の脆弱性を直接修正できる点でWAFとは異なる.

このフレームワークの機能は,Pythonを用いて実装された.
このフレームワークは,具体的には4つの工程によりプログラムを修正する.
まず活動中のオブジェクトから,アプリケーション開発者が実装した関数のソースコードを取得する.
このフレームワークは,コールバック関数を実行可能な形式で受け取り,その関数のソースコードを組み込み関数を利用してフレームワークに格納する.
次にそのソースコードを抽象構文木(AST)に変換する.
その後,そのASTを修正する.
そして最後に実行可能なオブジェクトに変換しなおす.
この工程により,アプリケーション開発者が実装した関数はフレームワークに修正されたのちに,フレームワークに格納される.

検証として,このWebアプリケーションフレームワークの機能を利用して,アプリケーション開発者が実装したコードに脆弱性があるWebアプリケーションを実装して評価を行った.
結果として,組み込み関数によってできたSQLiと認可制御の不備という二つの脆弱性を低減できることがわかった.

\section{提案手法}
この章ではコールバック関数を修正する機能を持つWebアプリケーションフレームワークを記述する.
このフレームワークが実行される時,4つの過程を経てコールバック関数が修正される.

まずはじめにフレームワークにコールバック関数とルーティングに必要な情報が格納される.
コールバック関数はルーティングに必要な情報とともに一つの辞書形式のデータになり,リストに格納される.
この時点では,コールバック関数は活動中のオブジェクト,つまりバイナリデータである.

次にコールバック関数を処理しやすくするために,格納されているコールバック関数を活動中オブジェクトからASTに変換する.
ASTはプログラムを実行可能なバイナリデータにするプロセスで得られる中間コードである.
ASTはノードで表され,各要素は意味のあるプログラムのコードオブジェクトを表している.
ASTの各ノードは意味のある名前または属性名を持っているため,バイナリを直接変更するよりも修正が容易である.
また,ASTはソースコードを修正するよりも修正が容易であるという利点がある.
フレームワーク内でコールバック関数を修正する際,Webアプリケーションフレームワークの開発者はアプリケーション開発者が実装するコールバック関数を直接見ることができない.
そのため,ソースコードを修正するために正規表現や文字列パターンを利用するのは困難である.
ASTのノードはソースコードから意味のない部分が削除されているため,解析と修正が容易である.

三つ目のステップとして,コールバック関数を修正する.
コールバック関数を修正する,脆弱性ハンドリング関数をフレームワーク内に実装する.
脆弱性ハンドリング関数はコールバック関数のASTを要素とするリストを引数として受け取り,再帰的に探索して脆弱性を検出する.
その後検出された脆弱性を修正してASTのリストを返す.
ASTのリストを受け取ることで,脆弱性ハンドリング関数はコールバック関数間の内容の違いから脆弱性を検出できるという利点がある.

最後の工程として,修正されたASTを活動中のオブジェクトに変換することで,リクエストに対して修正されたコールバック関数が呼び出されるようになる.

\section{実装}
この章では提案手法で示したコールバック関数を修正する4つの過程についての実装をそれぞれ記述する.
一つ目が,コールバック関数を受け取り,フレームワークに格納することである.
コールバック関数を受け取る際に,フレームワークはパスとメソッドを受け取る.
これらのパスとメソッドの情報を元にコールバック関数は呼び出される.
このフレームワークはパス,メソッド,コールバック関数を一つにまとめた辞書形式のデータにし,一つにまとめられたデータを要素としてリストに格納する.

二つ目に,コールバック関数をASTに変換する.
Pythonが提供しているASTに対するヘルパーであるastモジュールが,実行可能なバイナリデータを直接ASTに変換できないため,コードオブジェクトからソースコードを取得する.
コードオブジェクトからテキストを受け取るために,inspectモジュールを利用する.
inspectモジュールは,活動中のオブジェクトから情報を取り出すモジュールである.
このモジュールのうち,getsourceメソッドを利用して関数のソースコードを取得する.
getsourceメソッドは,ソースコードの該当部分をテキストファイルとして読み込むことで,ソースコードを取得する.
動的に実行したコードオブジェクトから,ソースコードを取得しようとするとOSErrorを起こすことがあるため注意が必要である.
特に,フレームワークによって修正された関数は動的に生成された関数であるため,ソースコードを取得できない.
その後,Pythonのastモジュールのparseメソッドを利用してソースコードをastに変換する.

%ASTを解析,修正する
三つ目はコールバック関数のASTを解析し修正する工程である.
コールバック関数を解析・修正する脆弱性ハンドリング関数は複数あり,リスト形式で格納されている.
脆弱性ハンドリング関数が複数あるのは,脆弱性ハンドリング関数の保守性を高めるためである.
コールバック関数のASTはそれぞれの脆弱性ハンドリング関数によって解析・修正される.
フレームワークは脆弱性ハンドリング関数をリストから一つずつ取り出しそれぞれを実行する.
脆弱性ハンドリング関数の引数はASTに変換されたコールバック関数のリストであり,解析・修正されたASTのリストが脆弱性ハンドリング関数の戻り値である.

%ASTからコードオブジェクトを作り,格納する
最後に修正されたASTを活動中のオブジェクトに変換する.
PythonのASTを実行可能なオブジェクトに変換するexecという組み込み関数を利用する.
この関数を利用してできた関数はデフォルトでは,ローカル変数であるため注意が必要である.
以下の工程を経てコールバック関数は修正され,フレームワーク内に格納される.

\section{実験}
脆弱なコールバック関数を持つ簡易的なWebアプリケーションを,本報告におけるWebアプリケーションフレームワーク上に実装し実験を行った.
脆弱性はSQLiと認可制御の不備である.
それぞれの脆弱性とその対策ついて記述する.
このフレームワークの実装はPython3.7上で実装された.
実験を行った環境は,OSが MacでOS X El Capitan 10.11.6, CPU が Intel Core i5 (2.95GHz),メインメモリが8GBであった.

\subsection{SQLiの修正}
SQLiはデータベースを不正に利用される脆弱性である.
SQLiはデータベースに利用されているリクエストの一部が適切に無害化されていないという脆弱性であり,SQLi攻撃はこの脆弱性を利用して行われる脆弱性攻撃である.
悪意のあるリクエストをWebアプリケーションに送信することで,そのアプリケーションに接続されたデータベースを不正に動作させ,データを取得したりデータベースを改ざんしたりする攻撃である.

SQLiを修正する実験として,SQLi脆弱性があるコールバック関数を実装した.
脆弱性は具体的には以下のコードである.
\begin{lstlisting}[language=python]
cur.execute('{query}'.format(action=request.params["query"]))
\end{lstlisting}
cur.executeメソッドがSQLを実行するメソッドである.
request.paramsはリクエストパラメータである.
このソースコードはrequest.paramsを無害化していないため脆弱である.
この脆弱性を低減するために,cur.executeメソッドを検出し,テーブルを消去する"drop"命令とwhere句が含まれない"select"命令がクエリに挿入されていると,そのクエリを空文字のクエリに変換する脆弱性ハンドリング関数を実装した.

\subsection{認可制御の不備}
認可制御の不備は権限が必要なデータを権限を持たないユーザが閲覧可能である脆弱性である.
以下に脆弱なコールバック関数を示す.
\begin{lstlisting}
@app.route("^/login$", "POST")
def do_login(request):
    id = request.params["ID"]
    password = request.params["PASSWORD"]
    if is_admin(id, password):
        return "ADMIN_PAGE"
    else:
        return "LOGIN_PAGE"

@app.route("^/home$", "GET")
def home(request):
    return "ADMIN_PAGE"
\end{lstlisting}
/loginを要求すると呼び出されるコールバック関数であるdo\_login内のis\_admin関数はIDとパスワードにより認証を行っており,IDとパスワードを知っている管理者のみが"ADMIN\_PAGE"を閲覧可能になっている.
is\_admin関数はこのWebアプリケーションフレームワークの組み込み関数であり,IDとパスワードが正しければTrue,そうでなければFalseを返す.
このアプリケーションは/homeを要求すると権限を持っていないユーザも"ADMIN\_PAGE"を閲覧できてしまう点で脆弱である.
この対策として,is\_admin関数が条件である戻り値(上記のコードでは"ADMIN\_PAGE")のうち,is\_admin関数による認証がない戻り値に対して,自動で同様の条件を追加する脆弱性ハンドリング関数を実装した.

\section{結果}
二つの脆弱性を対策するためのそれぞれの脆弱性ハンドリング関数は実験における脆弱なコールバック関数をそれぞれ修正した.
SQLiに対してはdropを含むクエリが実行されていないことを確認した.
認可制御の不備では,/homeを要求すると,修正前は"ADMIN\_PAGE"に遷移されたが,脆弱性ハンドリング関数を実装した後は"LOGIN\_PAGE"ヘ遷移されることを確認した.
この二つの結果からこのフレームワークは脆弱性の影響を低減できることがわかった.

\section{まとめ}
本論文では,アプリケーションが持つべき機能として,コールバック関数を自動的に解析し,必要ならば修正する機能を提案し,実装したのちに評価した.
この実装と評価からSQLiと認可制御の不備を一部対策可能であるとわかった.
今後改良をしていくことで,このアプリケーションフレームワークを利用して,セキュアなアプリケーションを簡単に実装できるようになると考えている.

\begin{thebibliography}{3}
\bibitem{weinberger2011systematic}
J.~Weinberger, P.~Saxena, D.~Akhawe, M.~Finifter, R.~Shin, and D.~Song, ``A
  systematic analysis of xss sanitization in web application frameworks,'' in
  \emph{European Symposium on Research in Computer Security}.\hskip 1em plus
  0.5em minus 0.4em\relax Springer, 2011, pp. 150--171.

\bibitem{kruegel2003anomaly}
C.~Kruegel and G.~Vigna, ``Anomaly detection of web-based attacks,'' in
  \emph{Proceedings of the 10th ACM conference on Computer and communications
  security}, 2003, pp. 251--261.

\bibitem{epp2017anomaly}
N.~Epp, R.~Funk, C.~Cappo, and S.~Lorenzo-Paraguay, ``Anomaly-based web
  application firewall using http-specific features and one-class svm,'' in
  \emph{Workshop Regional de Seguran{\c{c}}a da Informa{\c{c}}{\~a}o e de
  Sistemas Computacionais}, 2017.

\end{thebibliography}

\end{document}
