%%
%% 研究報告用スイッチ
%% [techrep]
%%
%% 欧文表記無しのスイッチ(etitle,eabstractは任意)
%% [noauthor]
%%

%\documentclass[submit,techrep]{ipsj}
\documentclass[submit,techrep,noauthor]{ipsj}



\usepackage[dvips]{graphicx}
\usepackage{latexsym}
\usepackage{url}
\usepackage{listings}

\def\Underline{\setbox0\hbox\bgroup\let\\\endUnderline}
\def\endUnderline{\vphantom{y}\egroup\smash{\underline{\box0}}\\}
\def\|{\verb|}
%

%\setcounter{巻数}{59}%vol59=2018
%\setcounter{号数}{10}
%\setcounter{page}{1}


\begin{document}


\title{Webアプリケーションのための攻撃手法の収集\\
〜攻撃者の誘導手法の実装と評価〜 }

\etitle{Collecting of Attack Methods for Web Applications \\ (version 2018/10/29)}

\affiliate{KU1}{九州大学 工学部 電気情報工学科 計算機工学課程\\
Kyushu University, Fukuoka, 819--0385, Japan}

\affiliate{KU2}{九州大学 情報基盤研究開発センター\\
Kyushu University, Fukuoka, 819--0385, Japan}


\author{久保田 康平}{Kubota Kohei}{KU1}[1TE15122K@s.kyushu-u.ac.jp]
\author{小出 洋}{Koide Hiroshi}{KU2}[koide@cc.kyushu-u.ac.jp]

\begin{abstract}
本論文では, 不審な行動や特定のパスへのアクセスをしたユーザを攻撃者と見なし,
その攻撃を誘導することで攻撃手法の収集・監視することによるWebアプリケーションの保護を目的としている.
その目的を達成するために既存のWebアプリケーションを変更することなく高対話型ハニーポット機能の付加を行った.
また, 特定のアクセスを攻撃と推測し, 誘導・監視することを可能にした.
システムの設計と実装を行い, クラウド上で立ち上げて実際に監視した結果, 特定のアクセスに対する攻撃の検出を確認した.
\end{abstract}


%
\begin{jkeyword}
侵入検知・検出, Webセキュリティ
\end{jkeyword}

\begin{eabstract}
The objective of this papar is to protect Web Applications from attackers by collecting suspicious behavior and accesses to specific paths.
To achieve this objective, applications in operation was added to functions of honeypot without changing the applications.
Also, the applications were enabled to conducting and observing specific accesses that were regarded as attacks.
The system was designed, built, and located on a cloud service. Attacks that searched specific paths were consequently detected.
\end{eabstract}
%
\begin{ekeyword}
Attack Detection, Web Security
\end{ekeyword}

\maketitle

%1
\section{はじめに}
本研究では, 既存のアプリケーションの機能を変更することなく不審な行動の監視, phpMyAdmin\cite{label6}を狙った攻撃手法の収集・分析を行い, 具体的な攻撃手法の検出をすることでその対策を講じたり, 不正アクセスをハニーポットに誘導し被害を小さくしたりすることで, Webアプリケーションを保護することを目的とする. その目的を達成するために, 特定のサーバへの要求の際, 必ず経由するプロキシサーバであるリバースプロキシ\cite{label4}によってWebアプリケーションに擬似APIを付加し, アプリケーションにハニーポット機能を持たせること, またURLにおいてphpMyAdminという文字列を検出した際, ハニーポットに誘導することで攻撃者の動向を監視するシステムを設計・実装した. また実際にクラウド上に設置し, アプリケーションに接続したユーザのリクエストの中でシステムが攻撃と推測したものを監視した.
URL上のphpMyAdminという文字列を攻撃として検出する理由として, \cite{label2}において攻撃者の多くがphpMyAdminのディレクトリ及びsetup.phpに対しての接続が可能であるか調べていたことがあげられる. その研究において実装したシステムはユーザがアプリケーション内に存在しないリクエストを規定回数以上行った時に, そのユーザを攻撃者とみなし監視するシステムであったが, 攻撃者の90\%がphpMyAdminへのアクセスを試みようとしていることがわかった. したがって, これらのURLを検索をあらかじめ攻撃とみなすことで攻撃者を早期に検知することができると考えた. \par
また本研究において, Webアプリケーションの保護を達成するために,既存のアプリケーションに対してハニーポット機能を容易に付加することを目標としている. 容易にハニーポット機能の付加を可能にすることで, アプリケーションを作成するユーザはハニーポット機能を簡単に付加できるようになりWebにおけるセキュリティを高めることが可能になる. \cite{label2}における実装では, 擬似APIの付加によって攻撃者を擬似データベースに誘導している.
擬似データベースには, 運用しているアプリケーションで用いているデータベースと同じテーブルを用意しているが正規データベースのデータは入っていないため, 高対話型ハニーポットでありながら, 低対話型ハニーポットと同程度の安全性を担保している. この際, 正規APIを継承したクラスである擬似APIを作成することで, 攻撃者を擬似データベースに誘導するが, そのためにはハニーポット化するアプリケーションのAPIを理解する必要があり難易度が上がる.
このことを解決する方法の一つとして, アプリケーションにあらかじめ擬似データベースのみを接続しておくことでAPIを変化することなく, ハニーポット機能を付加することが可能になった.\par
この実装によってアプリケーションを変更せずにphpMyAdminを文字列に含むリクエストを書き換えるようなリバースプロキシの変更のみによってphpMyAdminをハニーポット化することが可能という結果が得られた. またphpMyAdminへの攻撃誘導を付加したシステムをクラウド上に設置し監視した結果, アプリケーションへの不審な行動を検出し記録することが可能だということ, 不審な行動によるハニーポットへの誘導及び記録モードへの移行よりも早期に, phpMyAdminのsetup.phpを狙った脆弱性攻撃を検出することが可能だという結果が得られた.

%2
\section{システムの機能と手法・実装}
本章では実装のために用いたシステムの機能について記述するとともに, 実装におけるネットワークの設計のための手法と実際に行った実装について説明する.

%2.1
\subsection{システム構成}
本研究における提案手法では, 攻撃検出を行うリバースプロキシ機能を搭載したシステムと収集した情報の閲覧・分析を行う管理システムの2種類から構成される. 攻撃検出システムで収集された情報はデータベースに記録され, 管理システム上で確認することが可能である. 管理システムのデータベースはセッション情報とリクエスト情報を格納するRDBと接続先情報のキャッシュとPOSTメソッドの場合のパラメータを格納するNoSQL DBで構成されている. コンテンツは最低1つでも動作するが, 複数個設置することで攻撃者に1つのコンテンツがハニーポットと判断された場合, 再接続の際に, 他のコンテンツに接続させて引き続き攻撃活動を誘導することが可能となる. これは, 接続状態のキャッシュが無効になった際, 接続を行うと初回接続と同様とみなして通信を行うためである. また,攻撃検出システムのリバースプロキシ機能により, コンテンツに外部からの接続を行うことは不可能であり, 攻撃検出システムを介してのみ接続可能である.

%2.2
\subsection{システムの機能}
ここでは, システムの設計のために用いた情報システムについて, その機能を実装する.

%2.2.1
\subsubsection{リバースプロキシ}
リバースプロキシとは, 特定のサーバへの要求を必ず経由するように設置されたプロキシサーバのことである. リバースプロキシを使用する利点は以下のようなものがあげられる.
リバースプロキシを前置することで, 内部ネットワークに存在するWebサーバを外部から隠蔽し安全性を高めることができる.また複数のサーバ群に処理を分散させることで付加分散を可能にすることである.
加えて, Webサーバ群にリバースプロキシを組み合わせる場合, リバースプロキシが各WebページにあるURLを書き換えることによって, 異なるサーバ群を一つのアプリケーションサーバのように独立のサーバであるため, アプリケーションサーバ内を変更することはない.


%2.2.2
\subsubsection{ハニーポット}
ハニーポットは未対策のアプリケーションやOSに存在する脆弱性をシステムに用意し, 攻撃者をシステムに誘導することで, 攻撃活動や不正アクセスを促すシステムである. ハニーポットを用いることで, 攻撃者の行動や攻撃手法といった情報を収集することが可能である.
ハニーポットには, 攻撃者からの接続によって初めてハニーポットとしての役割を持つ対話型ハニーポットがある. そして対話型ハニーポットには低対話型ハニーポットと高対話型ハニーポットの2種類の分類がある.\par
低対話型ハニーポットは特定のプロトコルや脆弱性を持ったアプリケーションを擬似的に模倣することで, 攻撃者をエミュレーション環境に誘導し攻撃活動を促すハニーポットである. 低対話型ハニーポットの特徴として, 攻撃者が接続したシステム上で行っている動作やコマンドはシステム側でエミュレートされているものであるということが挙げられる.攻撃者はシステムに用意された疑似コマンドを利用するため,攻撃者側の画面上では攻撃が成功しているように見えているが,実際には実行されていないコマンドのためホストマシンへの影響はない.\par
一方, 高対話型ハニーポットは実際の物理マシンや仮想マシン上に脆弱性を持ったアプリケーションやOSを配置し動作させる事で, 攻撃者をシステムに接続させ攻撃活動を促すハニーポットである.高対話型ハニーポットは低対話型ハニーポットとは異なり,実際のマシン上でアプリケーションやOSの動作を行う.そのため,より高度な攻撃手法の取得が可能になる一方,OSコマンドやアプリケーションの全機能を利用できることから,ファイルシステムの破壊やハニーポット自体が踏み台とされるリスクもある.この事から高対話型ハニーポットを設置する際, 攻撃者が他のシステムに影響を与えないようにネットワークを考慮することが必要であるため, 運用が難しい.

%2.2.3
\subsubsection{phpMyAdmin}
phpMyAdminとは, MySQLサーバをウェブブラウザで管理するためのデータベース接続クライアントツールであり, PHPで実装されている. phpMyAdminを用いることで, SQL文を記述することなく, MySQLに対して様々な操作が行える. また, ユーザが任意のSQL文を記述して実行することもできる. このツールの特徴は直感的に扱えるWebインターフェイスであり, ほとんどのMySQL機能をサポートしており, クエリを使用することで複雑な問い合わせを作成することも可能であることである. 一方でphpMyAdminはWebサーバに設置するため, 容易にアクセスされてしまう危険性があり, また有名な管理ツールであるため攻撃の対象とされやすい.

%2.3
\subsection{先行研究における手法}
\cite{label2}では, 攻撃検出機能を搭載することで, 攻撃からWebアプリケーションを保護し, 攻撃手法を行うシステムをクラウド上に設置し監視していた. このシステムはHTTPリクエストを受け取った際にユーザ属性の状態に応じてコンテンツ側のAPIを正規と擬似に切り替えることで, Webアプリケーションを攻撃から保護することが可能であるものであった. また攻撃手法の収集も行っており, 攻撃傾向の分析も可能であった.\par
各システム及びコンテンツはLinuxコンテナ技術であるDockerを用いたものである. Docker\cite{label7}はDocker社が開発したコンテナ型の仮想化環境を提供するオープンソースソフトウェアである. コンテナ型はホストOS上に仮想化されたハードウェア上で動作するハイパーバイザ型やホスト型とは異なり, プロセス環境上においてファイルシステムやネットワークインターフェース, 名前空間を独立するようにさせて動作させるものである. ホストOSの1プロセスとして動作するため, 軽量であり, 特定の環境に依存することなく動作することが可能である.\par
Dockerを用いることで, 各コンテナ内で実行されているプロセスはそれぞれ内部で完結しており, 他のコンテナに対して影響を与えない. そのため, システムは高対話型ハニーポットの機能を有しているが, 低対話型のハニーポットと同様の安全性を担保している. \par
\cite{label2}におけるシステムは攻撃検出を行うリバースプロキシ機能を搭載した攻撃検出システムと収集した情報の閲覧・分析を行う管理システムから構成される. 管理システムと攻撃検出システムは分離して動作させることも可能であるため, 攻撃検出システムのみを搭載した端末とそれらの情報を集約する管理システムに分けて運用することも可能である. 攻撃検出システムで収集した情報は管理システムのデータベース上に記録されており, 管理システム上で確認することが可能である. 管理システムのデータベースはセッション情報とリクエスト情報を格納するRDBと接続情報のキャッシュとPOSTメソッドの場合のパラメータを格納するNoSQL DBで構成されている.
ユーザがこのシステムに接続を行う場合, 攻撃検出システムを介して各コンテンツに接続を行うようになっている. その際に行う処理の流れは以下のようなものである.\par
\vspace{1pc}
1. 接続先情報の読み出し\par
2. コンテンツと接続元IPアドレスの紐付け(初回接続時のみ)\par
3. HTTPリクエストの読み取り\par
4. セッション情報の記録\par
5. HTTPリクエスト情報の記録・書き換え\par
6. コンテンツにHTTPリクエストを送信\par
7. ユーザにHTTPレスポンスを送信\par
\par
\vspace{1pc}
ユーザが初めてシステムに接続すると, 攻撃検出システムは接続元IPアドレスをキーとしてランダムに決定されたコンテンツを紐付けし, セッション情報と共にJSONでキャッシュを行う. セッション情報を識別する値はUUID Version4\cite{label14}によりランダムに生成される. UUID V4とはUniversity Unique Identifer Version4の略で128ビットの数値で表され, 一般的に16進法で表現される一意識別可能な識別子のことである. セッションのIDを生成した後, コンテンツ情報はシステムに登録されたものからランダムに決定してそのポート番号を格納する.
\par
接続先情報が読み出されたあと, HTTPリクエストの読み取りを行い, セッション情報を管理データベースに記録する. HTTPリクエストのあと, ユーザの状態に応じて情報の記録・HTTPリクエストの書き換えを行う. ユーザ属性の状態は通常モードと記録モードの2種類ある. 通常モードは, 一般ユーザがシステムに対して接続を行う際のモードであり, 一般的には通常モードが使用される. 記録モードは, ユーザがシステムに対して攻撃を続けた際に用いるモードであり, 通常モードから切り替えられ攻撃情報の収集を行うモードである. ユーザモードの切り替えはモードスコアの値により判定を行う. モードスコアの値はHTTPレスポンスを確認し, 認証が必要なページに認証なしで接続する場合に返される403や, ファイルが存在しない場合に返される404などといった場合に増加する. これは, 通常利用では起こりえないHTTPステータスコードが生じることで, 攻撃の疑いがあると推測を行っているためである. また, 切り替えの基準となるモードスコアの値は自由に変更することが可能であり, 誤検知が多くなった場合でも柔軟な運用が可能である.\par
記録モードにおける通信情報は, ユーザが接続するたびにHTTPリクエストから情報の取得を行っており, その情報を管理データベースに記録している. HTTPリクエストのヘッダには最低, 接続メソッド, 接続パス, ホスト名, 接続する際に使用するハードウェアやソフトウェアの情報が記載されているユーザエージェント, どのようなデータを受け取りたいかを記すAccept情報が記載されている.\par
取得する情報としては, 接続パスと接続メソッドの2種類がある. 接続メソッドがPOSTメソッドの時はパラメータの情報も合わせて取得を行う. \par
コンテンツを受け取ったHTTPレスポンスはボディの情報をユーザが利用可能なポート番号の修正を行う必要がある. 例えば, ページ遷移時のリンクの情報がコンテンツから受け取った情報をそのままユーザに送信するとコンテンツのポートに対して接続を行い, 攻撃検出システムを介さないため, ユーザのクライアントからコンテンツに接続を行うことが不可能になる. ポート番号の修正に関しては, 正規表現を用いたコンテンツポート番号とマッチしたものを攻撃検出システムが用いているポート番号に書き換えて送信を行う. これにより, 攻撃検出システムを介して接続を行うことでユーザがコンテンツに接続することが可能となる.
コンテンツに接続されるデータベースは, 正規運用中の場合に用いるデータベースと擬似運用中の場合に用いるデータベースの2種類が接続されており, 正規APIは正規運用のデータベース, 擬似APIは擬似運用のデータベースに接続している. 2つのデータベースのテーブル構造にはどちらも同一の構造が格納されている. これは攻撃者がSQLインジェクションを行ってシステム側が記録モードに移行した際にテーブル構造の変化によりハニーポットだと判断されることを防ぐためである.\par
コンテンツのAPI作成時に攻撃検出システムと連動を行うために正規と擬似の2種類のAPIを作成する必要がある. その際に設定するルーティング情報は, 攻撃検出システム側でパス情報が変更された状態でコンテンツに送信されるためパスの統一が必要である. 以下はhomeというAPIを実行する場合の一例である.\par
正規APIと擬似APIは同一の動作を行うため, 同じデータ構造を用意する必要がある. 擬似API作成にはオブジェクト指向の手法を用いて正規APIを継承することにより, 作成することが可能である.
正規運用時に攻撃検出システムは通常モードのため, 正常通信としてシステムが認識しており, 受信したHTTPリクエストをコンテンツにそのまま送信している. 一方, 擬似運用時に攻撃検出システムは記録モードに移行しているため, すべての通信情報は記録している. またシステムは正規APIを擬似APIに書き換え, HTTPリクエストをコンテンツに送信している. 擬似APIに接続しているため読み出しを行うデータベースは擬似運用のデータベースが読み出される. これにより擬似運用時には正規運用のデータベースを読み出さないためデータベース内の情報の安全性を保つことができる.\par

%2.4
\subsection{追加した手法とその実装}
リバースプロキシで受け取ったリクエストはリバースプロキシ内で必要に応じて変換される. 例えば上述のシステムにおけるリクエストにおいては, ユーザがリバースプロキシであるポートにHTTPリクエストを送ると, アプリケーションサーバへのポートへとリクエストが変更される. また攻撃者だと判断された場合も, リクエストの書き換えが行われ, それによりAPIが変更されるようになっている. 
このリクエストパスの書き換えを行うことでユーザが送ったHTTPリクエストを書き換えて, 異なるアプリケーションを利用させることが可能になっている.
この機能を用いてDockerで8880番ポートにphpMyAdminのコンテナを作成し, リバースプロキシとデータベースに接続する. ユーザはリバースプロキシのIPアドレス:リバースプロキシのポート番号/(phpmyadminもしくはphpMyAdmin)を検索するとリバースプロキシはHTTPリクエストの書き換えを行いHTTPメソッドの, /(phpmyadminもしくはphpMyAdmin)を消し, リクエストヘッダのHost部分をphpMyAdminのポートである8880番ポートへと変更するようにする. またphpMyAdminの表示のためのソースも, 同様にポートを変更することで正しく表示するようにする.
リクエストにphpMyAdminもしくはphpmyadminという文字列がある場合, リクエストを変更する関数を導入し, 転送するポートと変更したリクエストを戻り値とする. POSTメソッドの場合, クエリ文字列をリクエスト文字列の最後尾に付加させ, GETメソッドに変更することでリクエストを受け取るようにしている.\par
この実装を行う際, phpMyAdminは高対話型ハニーポットとして動作させるため, phpMyAdminに接続させるデータベースは擬似データベースのみとなっている. このハニーポット化は擬似データベースにしか接続していないため, 擬似APIを作成する必要がない. つまりアプリケーション自体の変更を加えることなくハニーポット機能を付加することが可能である.
また, 擬似データベースには既存のアプリケーションに用いるテーブルと同様のテーブルは存在するが正規のデータベースのデータが入っていないため, 情報が漏洩することがなく安全性を確保できる. \par
この実装における利点は, 既存のアプリケーションに追加しやすい点とハニーポット化されたアプリケーションを並列して設置することが可能で廃棄しやすい点である. 既存のアプリケーションにリバースプロキシを接続し, そのリバースプロキシにハニーポット化されたphpMyAdminを接続すれば高対話型ハニーポットとしての機能をアプリケーションに付加することが可能になる. また\cite{label11}において自動化されたツールを用いた攻撃の収集を向上させる手法として, 実在しないパス名がURLに記述した場合それを実在するURLに変換することで攻撃を誘導する手法がある.
しかし, その手法において実在しないパスが, 実在するパスと大きく異なると実在するパスに変換できないという結果が得られている. この際, 変換できなかったパスを元にDockerコンテナを用いてハニーポット化されたアプリケーションを追加することで攻撃検出精度を向上できる可能性があり, その実装はコンテナを立ち上げてリバースプロキシに接続するという方法で実現でき, さほど難しくない. また攻撃等でアプリケーションとしての機能を保てなくなった場合, コンテナを廃棄, 再構築することでアプリケーションの機能を簡単に復元可能である.




%3
\section{実験}
本研究で実装したシステムをクラウド上に設置して実験を行った. 検証環境はAmazon Web Service(AWS)\cite{label9}のクラウド上である. Amazon Web Serviceは, Amazon.comが提供しているクラウドコンピューティングサービスである. AWSのサービスの一つであるAmazon Elastic Compute Cloud(EC2)\cite{label9}は, アマゾンが提供する計算資源を用いてアプリケーションを実行するAWSのクラウド内で提供しているウェブサービスであり, インスタンスと呼ばれる仮想コンピューティング環境を提供している.
このAWS上に追加実装前のシステムと追加実装後のシステムをそれぞれ一つずつ, それぞれ2019年1月23日から2019年1月29日までの7日間設置した. この時のAWSのインスタンスはT2microを利用した. インスタンスの性能を以下に示す.
\vspace{1pc}
\begin{table}[h]
\centering
\begin{tabular}{|c|c|}
\hline
インスタンス & T2micro \\
\hline
vCPU & 1 \\
\hline
メモリ[GB] & 1 \\
\hline
ストレージ & EBSのみ \\
\hline
\end{tabular}
\caption{インスタンスの性能}
引用元 \url{https://aws.amazon.com/jp/ec2/instance-types/}
\end{table}

%4
\section{結果}

%4.1
\subsection{攻撃検出件数の比較}
追加実装を行う以前のシステムへのユーザへのアクセス件数は368件あり, そのうち1件が攻撃だと判断され全体の0.2\%が記録モードに移行した. 一方, 追加実装を行ったシステムにおいてはユーザのアクセスが 185件あり, そのうちアクセスの 0.5\%である1件が記録モードに移行した.

%4.2
\subsection{phpMyAdminのアクセスによる攻撃検出}
追加実装を行った攻撃検出システムにおいてsetup.phpに対しての接続が確認された.
調べたところ\cite{label10}, この攻撃はphpMyAdminのバージョンが3.3.10.2未満, もしくは3.4.3.1未満でphpMyAdmin配下にsetup/index.phpとsetup/config.phpが存在し, 外部から接続できる時に可能な脆弱性攻撃であることがわかった.
本研究で用いたphpMyAdminはバージョンが4.8.4であり, この攻撃への対処があらかじめされていたことからアクセスを試行する以上の攻撃には発展しなかった.

%5
\section{考察}

%5.1
\subsection{攻撃検出件数}
追加実装を行ったユーザのアクセス件数は185件, 追加実装を行っていない時のユーザのアクセス件数は386件であった. 一方で攻撃検出件数は追加実装を行った場合と行っていない場合でそれぞれ1件ずつ確認した.
しかし, 1週間と短い期間での実施であり, 今後インスタンスの数を増やしたりより長い期間監視を行ったりすることで, 攻撃検出精度の変化をより詳細に判断可能になると考えられる.

%5.2
\subsection{phpMyAdminのsetup.phpを狙った攻撃の観測}
phpMyAdminを追加実装したシステムにおいてphpMyAdminのsetup.phpを狙った攻撃と見られる動向を観測した. 本研究で用いたphpMyAdminはバージョンが4.8.4であり, この攻撃への対処があらかじめされていたことからアクセスを試行する以上の攻撃には発展しなかったが, この攻撃を早期に検出したことで, 攻撃検出の精度の向上という点において有用性があると考えられる.\par
一方で, 具体的な攻撃手法を収集できなかったことは課題である. この課題を解決するために, システム内に脆弱性があるように見せかける脆弱性ハンドラを導入することでより詳細な攻撃を収集を可能にすると考えられる. 脆弱性ハンドラとは攻撃を検出した際, その脆弱性が存在するかのようにレスポンスを返すことで, 攻撃を促すシステムである.

%6
\section{まとめと今後の課題}

%6.1
\subsection{まとめ}
本研究において既存のWebアプリケーションのシステムを変更することなくハニーポット機能を付加し, リバースプロキシを用いて攻撃者をハニーポット化したphpMyAdminに誘導することが可能という結果が得られた. またphpMyAdminを文字列に含むリクエストを用いた攻撃検出手法は, 既存のアプリケーションを全く変更することなく, ハニーポット機能の付加が可能という結果が得られた. また, 今回追加実装したphpMyAdminはDockerを用いて起動しており, コンテナ内のプロセスは独立しているで他のコンテナに対して影響を与えない. したがって低対話型ハニーポットと同程度の安全性を担保できる.
またコンテンツ等が実際に稼働している高対話型のハニーポットであるが, 攻撃活動により稼働が難しい場合, コンテナの破棄・再起動が容易である.\par
機能上攻撃検出が可能だということがわかった一方で, 攻撃検出件数そのものが1件と少なく, 未知の攻撃に対する攻撃収集をすることもできなかった. そこで攻撃の検出を継続して行うとともに, 機能の改良を行う必要がある. 以下のような改良が今後の課題となる.

%6.2
\subsection{今後の課題}

%6.2.1
\subsubsection{レスポンスボディの書き換えによるページ転送の自動化}
今回提案した手法では, phpMyAdminのレスポンスボディでリクエストされるAPI1つ1つを既存のアプリケーションと比較して, ポートの変更を行うことでphpMyAdminのすべてのAPIが転送されるようになっている. しかしそれでは, 複数のアプリケーションをハニーポット化する際に実装が難しくなってしまう. レスポンスボディの書き換えを自動的に行えるようにすることで複数のハニーポット化するアプリケーションを並行して設置することが可能になると期待できる.

%6.2.2
\subsubsection{攻撃検出後のレスポンスコードステータス変換}
今回の提案手法において, 攻撃の検出は行えたが具体的な攻撃の手法を収集するには至らなかった. しかし攻撃者はクローリングすることで既知の脆弱性を検知しそれに基づき攻撃する. したがって攻撃の具体的な手法を検出を可能にするために, 攻撃を検出した時点での400番台のレスポンスコードを200番に変換する機能を追加することで具体的な攻撃を検出可能になることが期待できる.

%6.2.3
\subsubsection{提案手法における有用性評価の継続}
提案手法による攻撃検出システムは, 現在AWS\cite{label9}上のクラウド環境上に設置されている. しかしこのシステムにおいて, 攻撃手法の具体的な検出に至っていない. そこで上記のような機能の実装をし, 改良を行いながら有用性の評価を行う.\\
\vspace{1pc}
\noindent
\begin{acknowledgment}
本研究は、国立研究開発法人科学技術振興機構(JST) 戦略的国際共同研究プログラム(SICORP)およびJSPS科研費 JP18K11295の助成を受けたものである。
\end{acknowledgment}


\begin{thebibliography}{16}
\bibitem{label1} 警察庁: 平成29年中におけるサイバー空間をめぐる脅威の情勢等について, pp. 1\{9 (2018),\par
\url{https://www.npa.go.jp/publications/statistics/cybersecurity/data/H29_cyber_jousei.pdf}
\bibitem{label2} 野見山賢人,小出 洋: Webアプリケーションのための攻撃検出・防御システムHoppinの設計・実装,第117回プログラミング研究会  (2018).
\bibitem{label3} Nginx Inc.: Nginx, \url{https://nginx.org/ja/}
\bibitem{label4} 久保達彦, 道井俊介: nginx実践入門, 技術評論社, pp. 124 (2016).
\bibitem{label5} 山本健太, 齊藤泰一: Linuxコンテナ技術を利用したSSHハニーポットの提案と評価. コンピュータセキュリティシンポジウム2016 論文集, Vol. 2016, No. 2, pp.770\{776 (2016).
\bibitem{label6} phpMyAdmin contributors: phpMyAdmin, \par
\url{https://www.phpmyadmin.net/}
\bibitem{label7} Docker Inc.:Docker - Enterprise Container Platform,\par
\url{https://www.docker.com}
\bibitem{label8} Trend Micro Inc.: 標的型サイバー攻撃 \textbar トレンドマイクロ - Trend Micro\par
\url{https://www.trendmicro.com/ja_jp/security-intelligence/research-reports/threat-solution/apt.html}
\bibitem{label9} Amazon Web Services, Inc.: Amazon Web Services(AWS)-Cloud Computing Services, \par
\url{https://aws.amazon.com}
\bibitem{label10} LAC Inc.: phpMyAdminの脆弱性を狙った攻撃について \textbar セキュリティのラック, \par
\url{https://www.lac.co.jp/lacwatch/alert/20111216_000146.html}
\bibitem{label11} 八木 毅, 谷本 直人, 針生 剛男, 伊藤 光恭: 高対話型Webハニーポットにおける攻撃情報収集方式の改善. (2011),
\url{https://ipsj.ixsq.nii.ac.jp/ej/?action=repository_uri&item_id=74936&file_id=1&file_no=1}
\bibitem{label12} 澁谷芳洋, 小池英樹, 高田哲司, 安村通晃, 石井威望: 高対話型おとりシステムの運用経験に関する考察. 情報処理学会論文誌, Vol. 45 No.8. (2004). \url{https://ipsj.ixsq.nii.ac.jp/ej/?action=repository_action_common_download&item_id=10839&item_no=1&attribute_id=1&file_no=1}
\bibitem{label13} T ishikawa and K sakurai: Web Application Deception Proxy, Proc. IMCOM'17, pp. 74:1\{74:9 (2017).
\bibitem{label14} ta\_ta\_ta\_miya: UUID(v4)がぶつかる可能性を考えなくていい理由 - Qiita, \par
\url{https://qiita.com/ta_ta_ta_miya/items/1f8f71db3c1bf2dfb7ea}
\bibitem{label15} N Provos, D McNamee, P Macrommatis, K Wang and N Modadugu: The ghost in the browser analysis of web-based malware, HotBots '07 (2007).
\bibitem{label16} :水飲み場攻撃
\end{thebibliography}


\end{document}
