\documentclass[a4j,11pt]{jreport}
\usepackage{resume}

\title{Webアプリケーションを安全に\\するフレームワークの新しい機能}
\subtitle{修士論文試問予稿}
\author{久保田 康平}
\date{2021年2月17日}
\time{11:40〜12:00}
\professor{小出 洋 教授}
\location{オンライン(Microsoft Teams)}

\university{九州大学大学院}
\department{システム情報科学府}
\major{情報知能工学専攻}

\begin{document}
\maketitle

本研究は,Webアプリケーションのセキュリティ機能向上を目的にしている.
そのため本研究では,Webアプリケーション開発者が実装するコードを実行時に自動的に解析し,必要ならば修正する機能をWebアプリケーションフレームワークに持たせることを提案し,実装して評価を行う.

Webアプリケーション開発者は効率よく安全にWebアプリケーションを開発するためにWebアプリケーションフレームワークを利用する.
Webアプリケーションフレームワークは,Webアプリケーション開発において,利用することが多いメソッドをまとめたライブラリである.
セキュリティ面において,Webアプリケーションフレームワークが提供するメソッドは脆弱性対策がなされているものが多い.
したがって,Webアプリケーションフレームワークを利用した方が,利用しない時と比較して効率的にセキュアなWebアプリケーションを開発しやすい.
一方で,開発者は常に完全にセキュアなコードを書くことはできないため,Webアプリケーションフレームワークを利用して,脆弱性があるWebアプリケーションを実装してしまうことがある.
Webアプリケーション開発者が,フレームワークのメソッドが持つセキュリティ機能を正しく理解していなかったり,セキュリティ機能を持つメソッドを知らなかったりするからである.

この問題に対して本研究では,Webアプリケーション開発者が実装したソースコードを修正する機能を持つWebアプリケーションフレームワークを提案する.

提案手法を実証し評価を行った結果,この機能は実装されたコードの脆弱性を一部修正でき,実効開始時のオーバーヘッドは提案手法を適用しなかった場合とほとんど変わらないことを確認した.
実装された修正関数の蓄積は将来のアプリケーションのセキュリティの向上に寄与できるものである.

\preseninfo
\end{document}
