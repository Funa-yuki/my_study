\documentclass[a4paper,12pt]{jreport}
\usepackage{jgraduate}

\title{Webアプリケーションを安全に\\するフレームワークの新しい機能}
\author{久保田 康平}
\university{九州大学}
\department{大学院システム情報科学府}
\major{情報知能工学専攻}

\date{\today}

\begin{document}

\maketitle

\begin{abstract}
本論文は,Webアプリケーションのセキュリティ機能向上を目的にしている.
そのために本論文では,Webアプリケーション開発者が実装するコードを実行時に自動的に解析し,必要ならば修正する機能をWebアプリケーションフレームワークに持たせることを提案し,実装して評価を行う.
Webアプリケーションはインターネットを通して世界中から誰でも接続でき,対話的に通信できるという特徴から様々な攻撃の対象になる.
また,インターネットの普及に伴いWebアプリケーションの重要性は増し,同様にWebアプリケーションの防御もまた重要になっている.
脆弱性攻撃は,Webアプリケーションの設計上の欠点や仕様上の問題点である脆弱性を利用する攻撃である.
脆弱性攻撃の対策の一つは,Webアプリケーションに脆弱性を作らないことであり,そのためWebアプリケーション開発者はWebアプリケーションフレームワークを利用することがある.
Webアプリケーションフレームワークは,Webアプリケーション開発において利用することが多いメソッドを持つライブラリである.
それらのメソッドを利用することで効率よくアプリケーションを開発することができる.
セキュリティ面において,Webアプリケーションフレームワークが提供するメソッドは脆弱性対策がなされているものが多い.
したがって,Webアプリケーションフレームワークを利用した方が,利用しない時と比較して効率的にセキュアなWebアプリケーションを開発しやすい.
一方で,開発者は常に完全にセキュアなコードを書くことはできないため,Webアプリケーションフレームワークを利用して,脆弱性があるWebアプリケーションを実装してしまうことがある.
その理由の一つが,Webアプリケーション開発者がWebアプリケーションフレームワークを適切に利用できないことである.
Webアプリケーション開発者が,フレームワークのメソッドが持つセキュリティ機能を正しく理解していなかったり,
セキュリティ機能を持つメソッドを知らなかったりすることによって脆弱なWebアプリケーションが実装される.
この問題に対して本論文では,Webアプリケーション開発者が実装したソースコードを修正する機能を持つWebアプリケーションフレームワークを提案する.
提案手法を実証し評価を行った結果,この機能は実装されたコードの脆弱性を一部修正でき,レスポンスタイムは提案手法を適用しなかった場合とほとんど変わらないことを確認した.
実装された修正関数の蓄積は将来のアプリケーションのセキュリティの向上に寄与できるものである.
\end{abstract}

\tableofcontents
\listoffigures
\listoftables
\newpage
\pagenumbering{arabic}

\chapter{はじめに}
%目的・提案手法
本論文は,Webアプリケーションのセキュリティ機能向上を目的にしている.
その目的の達成のために,アプリケーション開発者が実装したプログラム中の関数や引数を解析し,実行時にその関数に脆弱性があった時には修正することができるWebアプリケーションフレームワークを提案,実装し評価する.

%背景(他との比較を少し入れるwaf,フレームワークのサニタイズ)
%Webアプリケーションに対する防御手法
Webアプリケーションセキュリティは,セキュリティ分野において重要である.
インターネットの普及に伴い,Webシステムは様々な場所や階層において様々な攻撃にさらされている.
Webシステムへの攻撃のうちアプリケーション層への攻撃の多くはアプリケーションのプログラムが持つ論理的な問題が原因である.
そのためWebアプリケーション開発者は攻撃を回避するために,アプリケーションの論理的な問題や設計上の欠点である脆弱性を作らない実装をする必要がある.
一方で,Webアプリケーション開発者は常にセキュアなコードを記述することはできず,脆弱性を残す実装をすることがある.
加えてWebアプリケーション層にはセキュリティに関するプロトコルや標準的な仕様がないため,Webアプリケーションの安全性は,Webアプリケーション開発者のセキュリティに関する知識や技術に依存する.
これらのWebアプリケーションの問題を解決しセキュリティを向上するために,Webアプリケーションの自動防御手法としてWebアプリケーションファイアウォール(WAF)やWebアプリケーションフレームワークの利用などが検討されている.

%Webアプリケーションファイアウォール
WAFは,Webアプリケーションを脆弱性攻撃から保護するためのシステムである.
WAFはWebアプリケーションとクライアントの間に配置され,クライアントからのリクエストを監視し,リクエストが攻撃リクエストかどうかを検証する機能を持つ.
攻撃を検出した場合,そのリクエストを遮断もしくは無毒化することで,Webアプリケーションへの攻撃の影響を低減する.
WAFはWebアプリケーションを修正することなく,脆弱性攻撃を低減することが可能であるため,アプリケーションを直接修正できない時に有効な対策である.
一方でWAFはアプリケーションを修正しないので,アプリケーション内の脆弱性を根本的に修正できないという欠点がある.
またWAFはアプリケーション内の論理的な設計や仕様を知らないため,一部の脆弱性を対策することが難しい.
WAFは通常,特殊文字を含むリクエストを攻撃として検出する.
したがって,リクエスト内に特殊文字を含まない攻撃をWAFが検出することは難しい.

%Webアプリケーションフレームワーク
Webアプリケーションフレームワークは,Webアプリケーションを効率よく開発するために,Web開発に多用される機能を関数やメソッドとして提供するライブラリである.
自動防御手法としては,クロスサイトスクリプティング(XSS)やSQLインジェクション(SQLi)のようなインジェクション攻撃に対する入力検証と自動サニタイズという機能を提供していることがある.
自動サニタイズとは,引数を検証し無毒化する機能である.


%提案手法
この問題を解決するために,本論文ではアプリケーション開発者が実装したソースコードを解析し,必要であれば修正するWebアプリケーションフレームワークを実装する.
%%具体的な方法
%%%%フレームワーク内に脆弱なソースコードと修正後のソースコードを定義しておく
%%%%開発者のソースコードを解析
%%%%修正から実行
%%貢献

%実装
%%4つのステップ
%%%%コールバック関数の格納

%評価
%%ローカルで攻撃
%%実効開始時の時間

%まとめ・考察


\chapter{背景}
\section{最初}

\chapter{関連研究}
\section{論文1}
\section{論文2}
\section{論文3}

\chapter{提案手法}
この章では,アプリケーションフレームワークの仕組みについての説明したのちに,本論文で提案するアプリケーションフレームワークがソースコードを修正する手法について記述する.
\section{アプリケーションフレームワーク}
アプリケーションフレームワークは

\section{ソースコードの修正}
ほげ

\chapter{実装}

\chapter{実験}

\chapter{結果}

\chapter{考察}

\chapter{おわりに}


\end{document}
